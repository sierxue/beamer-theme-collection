% !TEX program = xelatex
% 完整编译方法 2: xelatex -> bibtex -> xelatex -> xelatex
% Auriga theme
% https://github.com/anishathalye/auriga

\documentclass[14pt,aspectratio=169]{ctexbeamer}
\input{../_input-tex/ctexbeamer-settings.tex} % 中文字体设定
\usepackage{pgfpages}
\usepackage{fancyvrb}
\usepackage{tikz}
\usepackage{pgfplots}

\usetheme{auriga}
\usecolortheme{auriga}

% define some colors for a consistent theme across slides
\definecolor{red}{RGB}{181, 23, 0}
\definecolor{blue}{RGB}{0, 118, 186}
\definecolor{gray}{RGB}{146, 146, 146}

\title{标题}

\author{\underline{\bf\kai{张三}} \inst{1} \and {\bf\kai{李四}} \inst{2} \and {\bf\kai{王五}} \inst{3}}

\institute[shortinst]{\inst{1} 北京师范大学 \samelineand \inst{2}
四川联合大学 \samelineand \inst{3} 山东德州大学}

\begin{document}

{
  % rather than use the frame options [noframenumbering,plain], we make the
  % color match, so that the indicated page numbers match PDF page numbers
  \setbeamercolor{page number in head/foot}{fg=background canvas.bg}
  \begin{frame}
    \titlepage
  \end{frame}
}

% !TEX root = *-cn.tex

\begin{frame}{中文字体测试}
    \begin{itemize} % [<+- | alert@+>]
        \item 缺省字体,\textbackslash textbf \textbf{粗体},
            \textbackslash textit \textit{斜体}
        \item \textbackslash alert \alert{字体},
            \textbackslash emph \emph{强调字体}
        \item \textbackslash hei \hei{黑体},
            \textbackslash kai \kai{楷体},
            \textbackslash texttt \texttt{字体},
            \textbackslash textsf \textsf{字体}
        \item \textbackslash bfseries {\bfseries \kai{粗楷体}},
            \textbackslash textbf \textbf{\kai{粗楷体}}
    \end{itemize}
\end{frame}
 % 中文字体测试
\begin{frame}{定理环境中文或英文显示测试}

{\bf 注意}:如使用 *-cn-ctex.tex, 下面显示~{\bf 定理} 和~{\bf 证明};
如使用 *-cn-xeCJK.tex, 显示 Theorem 和 Proof.

\begin{theorem}
    不存在最大的质数。
\end{theorem}

\begin{proof}
    Suppose $p$ were the largest prime number.
    Let $q$ be the product of the first $p$ numbers, then $q + 1$ is not
    divisible by any of them.
    But $q + 1$ is greater than $1$, thus divisible by some prime number
    not in the first $p$ numbers.\qedhere
\end{proof}
\end{frame}
 % 定理环境中文或英文显示测试
\input{slides/bullets}
\input{slides/split}
\input{slides/figure}
\input{slides/centered}
\input{slides/monospace}
\input{slides/brackets}
\input{slides/link}

\end{document}
