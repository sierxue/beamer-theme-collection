% !TEX program = xelatex
% 完整编译方法 2: xelatex -> bibtex -> xelatex -> xelatex
% Auriga theme
% https://github.com/anishathalye/auriga

\documentclass[14pt,aspectratio=169]{ctexbeamer}

\setCJKmainfont[
    AutoFakeBold=true
]{FandolFang-Regular} % Beamer 默认使用 sf 字体族。

\setCJKsansfont[
    BoldFont=FandolSong-Bold,
    ItalicFont=FandolKai-Regular
]{FandolSong-Regular}

\setCJKmonofont[
    BoldFont=FandolHei-Bold
]{FandolHei-Regular}

\newCJKfontfamily\song[
    BoldFont=FandolSong-Bold
]{FandolSong-Regular}

\newCJKfontfamily\hei[
    BoldFont=FandolHei-Bold
]{FandolHei-Regular}

\newCJKfontfamily\kai[
    AutoFakeBold=true
]{FandolKai-Regular}

\newCJKfontfamily\fang[
    AutoFakeBold=true
]{FandolFang-Regular}
 % 中文字体设定

\usepackage{pgfpages}
\usepackage{fancyvrb}
\usepackage{tikz}
\usepackage{pgfplots}

\usetheme{auriga}
\usecolortheme{auriga}

% define some colors for a consistent theme across slides
\definecolor{red}{RGB}{181, 23, 0}
\definecolor{blue}{RGB}{0, 118, 186}
\definecolor{gray}{RGB}{146, 146, 146}

\input{../_input-tex/frame-title-chinese.tex} % 标题页

\begin{document}

{
  % rather than use the frame options [noframenumbering,plain], we make the
  % color match, so that the indicated page numbers match PDF page numbers
  \setbeamercolor{page number in head/foot}{fg=background canvas.bg}
  \begin{frame}
    \titlepage
  \end{frame}
}

\begin{frame}{中文字体测试}
    \begin{itemize} % [<+- | alert@+>]
        \item 缺省字体,
            \textbackslash textbf \textbf{粗体},
            \textbackslash textit \textit{斜体}
        \item \textbackslash alert \alert{警示字体},
            \textbackslash emph \emph{强调字体}
        \item \textbackslash textbf \{\textbackslash textrm\{\}\}
            \textbf{\textrm{粗仿宋}},
            \textbackslash textrm \textrm{仿宋}
        \item \textbackslash textbf \{\textbackslash texttt\{\}\}
            \textbf{\texttt{粗黑体}},
            \textbackslash texttt \texttt{黑体}
        \item \textbackslash textbf \textbf{\kai{粗楷体}},
            \textbackslash kai \kai{楷体}
    \end{itemize}
\end{frame}
 % 中文字体测试
\input{../_input-tex/frame-theorem.tex} % 定理环境中文或英文显示测试
\input{slides/bullets}
\input{slides/split}
\input{slides/figure}
\input{slides/centered}
\input{slides/monospace}
\input{slides/brackets}
\input{slides/link}

\end{document}
